\documentclass[aspectratio=169]{beamer}
\usepackage{hyperref}
\usepackage{listings}

% \usepackage{fontspec}
% \setmonofont{Consolas}

\definecolor{background}{RGB}{39, 40, 34}
\definecolor{string}{RGB}{230, 219, 116}
\definecolor{comment}{RGB}{117, 113, 94}
\definecolor{normal}{RGB}{248, 248, 242}
\definecolor{identifier}{RGB}{166, 226, 46}
\definecolor{keyword}{HTML}{F92672}
\definecolor{numbers}{HTML}{AE81FF}
\definecolor{types}{HTML}{66D9EF}


\lstset{
    language=C,
    tabsize=4, % tab space width
    showstringspaces=false, % don't mark spaces in strings
    numbers=left, % display line numbers on the left
    basicstyle=\ttfamily,
    numberstyle=\color{comment}\ttfamily, % Line numbers
    commentstyle=\color{comment}\ttfamily, % comment color
    otherkeywords={>,<,-,!,=,~}, % Color operators too
    morekeywords={>,<,-,!,=,~},
    keywordstyle=\color{keyword}\ttfamily, % keyword color
    stringstyle=\color{string}\ttfamily, % string color
    morecomment=[l][\color{keyword}]{\#},
    emph={int,char,long,float,double,unsigned,namespace,typename},
    emphstyle={\color{types}\ttfamily\textit},
    escapeinside={@!}{!@},
    captionpos=b
}

\newcommand{\code}{\texttt}
\newcommand{\fn}{\color{identifier}}

\usetheme{Warsaw}

\addtobeamertemplate{footnote}{\vspace{-6pt}\advance\hsize-0.5cm}{\vspace{6pt}}
\makeatletter
% Alternative A: footnote rule
\renewcommand*{\footnoterule}{\kern -3pt \hrule \@width 2in \kern 8.6pt}
% Alternative B: no footnote rule
% \renewcommand*{\footnoterule}{\kern 6pt}
\makeatother

\setbeamercolor{normal text}{fg=normal,bg=background}
\setbeamercolor{structure}{fg=normal}

\setbeamercolor{alerted text}{fg=red!85!black}

\setbeamercolor{item projected}{use=item,fg=background,bg=item.fg!35}

\setbeamercolor*{palette primary}{use=structure,fg=structure.fg}
\setbeamercolor*{palette secondary}{use=structure,fg=structure.fg!95!black}
\setbeamercolor*{palette tertiary}{use=structure,fg=structure.fg!90!black}
\setbeamercolor*{palette quaternary}{use=structure,fg=structure.fg!95!black,bg=black!80}

\setbeamercolor*{framesubtitle}{fg=normal}

\setbeamercolor*{block title}{parent=structure,bg=background}
\setbeamercolor*{block body}{fg=black,bg=background}
\setbeamercolor*{block title alerted}{parent=alerted text,bg=background}
\setbeamercolor*{block title example}{parent=example text,bg=background}

\title{Strings}
\subtitle{Interview Skills}
\author{Richard Morrill}
\institute{Fordham University CS Society}
\logo{\includegraphics[width=2cm]{Images/css_logo_color.png}}
\date{Wednesday, Febuary 12th 2020}

\begin{document}
\begin{frame}
\titlepage
\end{frame}
\begin{frame}
    \frametitle{General Info}
    \begin{itemize}
        \item The questions will be based on C-style strings.
        \begin{itemize}
            \item You'll have to deal with them at some point.
            \item It also means many skills will translate to arrays.
            \item If you want to use fancy C++ stuff feel free.
        \end{itemize}
        \pause
        \item Make sure you understand the given sample case, and try to come up
        with your own.
        \pause
        \item Go with the simplest solution you can think of, but keep
        efficiency in the back of your mind.
    \end{itemize}
\end{frame}
\begin{frame}
    \frametitle{Helpful Hints}
    \begin{itemize}
        \item Just an array of \texttt{char}.
        \item The string variable is really just a pointer to the first character.
        \item There is a NULL (\texttt{`\char`\\0'}) character at the end of every string.
        \item Use \texttt{strlen(str)} to find the length (Won't include NULL).
        \item \texttt{`a' - `a' == 0} and \texttt{`z' - `a' == 25} Use this
        information wisely.
        \item Don't bother implementing simple functions such as \texttt{swap(a, b)} or \texttt{max(a,b)}
        \item C-Strings always pass by reference (pointer).
    \end{itemize}
\end{frame}
\begin{frame}
    \frametitle{Problem 1: Reverse a String (5 minutes)}
    Use function signature \code{void reverse(char* str)}.

    \vfill{}
    Do I really have to spell this one out for you?
\end{frame}
\begin{frame}[fragile]
    \frametitle{Problem 1 solution}
    \begin{lstlisting}
void reverse(char* str) {
    // Why did I store this value in a variable?
    int n = strlen(str);
  
    for (int i = 0; i < n / 2; i++) {
        swap(str[i], str[n - i - 1]); 
    }
} 
    \end{lstlisting}
\end{frame}
\begin{frame}
    \frametitle{Problem 1.1: Palindrome (2 minutes)}
    Quick! Take your solution to problem 1 and make it tell me if a string is a palindrome.
    \vfill{}
    Examples of Palindromes: racecar tacocat mom
\end{frame}
\begin{frame}
    \frametitle{Problem 2: Longest Common Substring (12 minutes)}

    Given two strings, return the length of the longest common contiguous segment.
    \vfill{}
    Examples:
    \begin{itemize}
        \item \textcolor{red}{seminar seminar}ian $\rightarrow$ 7
        \item tho\textcolor{red}{mas} \textcolor{red}{mas}sachusets $\rightarrow$ 3
        \item abcaf\textcolor{red}{ledt} bbcazt\textcolor{red}{ledt}or $\rightarrow$ 4
    \end{itemize}
\end{frame}
\begin{frame}[fragile]
    \frametitle{Problem 2 Solution}
    \begin{lstlisting}[basicstyle=\tiny]
int LCS(char* x, char* y) { 
    int xlen = strlen(x);
    int ylen = strlen(y);
    
    int result = 0;
  
    for (int xstart = 0; xstart < xlen - 1; ++ xstart) {
        for (int ystart = 0; ystart < ylen - 1; ++ ystart) {
            int xi = xstart;
            int yi = ystart;

            int currLen = 0;
            while(xi < xlen && yi < ylen && x[xi] == y[yi]) {
                ++currLen;
                ++xi;
                ++yi;
            }

            if (currLen > result) {
                result = currLen;
            }
        }
    }
    return result;
}
    \end{lstlisting}
\end{frame}
\begin{frame}
    \frametitle{Follow Up for Problem 2}

    The solution I presented was incredibly inefficient.
    \vfill{}
    How could we make this more efficient?
\end{frame}
\begin{frame}
    \frametitle{Problem 3: Most Common Character (8 minutes)}

    Given an input string consisting only of lowercase letters,
    return the most commonly used character.

\end{frame}
\begin{frame}[fragile]
    \frametitle{Problem 3 Solution}
    \begin{lstlisting}[basicstyle=\tiny]
char mostCommon(char* str) {
    int n = strlen(str);
    int buff[26];
    buff[0] = 0;

    for (int i = 0; i < n; ++i) {
        ++buff[str[i] - 'a'];
    }
    int max = 0;
    char maxChar = '\0';
    for (int i = 0; i < 26; ++i) {
        if (buff[i] > max) {
            max = buff[i];
            maxChar = i + 'a';
        }
    }
    return maxChar;
} 
    \end{lstlisting}

\end{frame}
\begin{frame}
    \frametitle{Problem 4: Compression (12 minutes)}
    
    Given a string of lowercase letters, replace every repeating letter
    with that letter followed by the number of times it occurred.
    \vfill{}
    Examples
    \begin{itemize}
        \item aabccc $\rightarrow$ a2bc3
        \item fffffaaff $\rightarrow$ f5a2f2
    \end{itemize}

\end{frame}
\begin{frame}[fragile]
    \frametitle{Problem 4 Solution}
    \begin{lstlisting}[basicstyle=\tiny]
char* compress(char* input) {
    int inputLen = strlen(input);
    char* output = new char[inputLen];
    int outputi = 0;

    char currChar = input[0];
    int currCharCount = 0;
    for (int inputi = 0; inputi < inputLen; ++inputi) {
        if (input[inputi] == currChar) {
            ++currCharCount;
        } else {
            output[outputi] = currChar;
            ++outputi;

            currChar = input[inputi];
            if (currCharCount > 1) {
                output[outputi] = currCharCount + '0';
                ++outputi;
            }
            currCharCount = 1;
        }
    }
    output[outputi] = '\0';
    return output;
}
    \end{lstlisting}
\end{frame}
\end{document}

